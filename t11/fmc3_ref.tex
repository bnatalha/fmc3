% Autor(a): Natália Azevedo de Brito

\documentclass[11pt,a4paper]{article}

% Pacotes utilizados
\usepackage[portuguese]{babel}
\usepackage[utf8]{inputenc}
\usepackage{upgreek,booktabs,yhmath,bbm}
\usepackage{amssymb,latexsym}
\usepackage{mathtools}

\usepackage{xcolor}

\usepackage{enumerate,enumitem}

\usepackage{amsthm}
\theoremstyle{definition}
\newtheorem*{defin}{Definição}

\makeatletter
\def \equalsfill{$\m@th \mathord=\mkern-7mu
	\cleaders \hbox{$\!\mathord=\!$}\hfill \mkern-7mu \mathord=$}
\makeatother
\newcommand{\defeq}{\ensuremath{\stackrel{\text{\tiny def}}{\hbox{\equalsfill}}}}
\newcommand{\defi}{\ensuremath{\stackrel{\text{\tiny (i)}}{\hbox{\equalsfill}}}}
\newcommand{\defii}{\ensuremath{\stackrel{\text{\tiny (ii)}}{\hbox{\equalsfill}}}}

% Novos comandos ------------------------------------------------------------------
\newcommand{\prodlangsymbol}{\cdot} % símbolo usado para representar a operação de produto de linguagens
\newcommand{\strconcat}{*} % símbolo usado para representar a concatenção de strings
\newcommand{\prodlang}[2]{#1 \prodlangsymbol #2} % produto de linguagens
\newcommand{\emptystr}{\Lambda}
\newcommand{\lS}[1]{\mathsf{S}(#1)} % conjunto de todas as string dum alfabeto #1
\newcommand{\lconcatclosure}[1]{#1^*}
\newcommand{\lplusclosure}[1]{#1^+}

% --- conjuntos ---
\newcommand{\natset}{\mathbb{N}} % naturais
\newcommand{\natsetnozero}{\natset\setminus\lset{0}} % naturais sem zero
\renewcommand{\emptyset}{\varnothing} % conjunto vazio
\renewcommand{\lgroup}[1]{\langle #1 \rangle} % notação de grupo
\newcommand{\lpowerset}[1]{\mathsf{Pow}(#1)} % powerset
\newcommand{\lset}[1]{\{ #1 \}}
\newcommand{\lsetwithdef}[2]{\{ #1 : #2 \}}
\newcommand{\lbigunionsymbol}{\bigcup}
\newcommand{\lbigunion}[2]{\lbigunionsymbol_{#1}{#2}}

% --- demonstração ---
\newcommand \lexplanation[1]{[#1]}